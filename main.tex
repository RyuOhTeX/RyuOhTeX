\documentclass[a4paper, 10pt,]{article}


\usepackage{CJKutf8}
\usepackage[colorlinks=true, allcolors=blue]{hyperref}
\usepackage{listings}
\usepackage{tikz}
\usepackage{xcolor}
\usepackage{graphicx}
\definecolor{codegreen}{rgb}{0,0.6,0}
\definecolor{codegray}{rgb}{0.5,0.5,0.5}
\definecolor{codepurple}{rgb}{0.58,0,0.82}
\definecolor{backcolour}{rgb}{0.95,0.95,0.92}

\lstdefinestyle{mystyle}{
    backgroundcolor=\color{backcolour},
    commentstyle=\color{codegreen},
    keywordstyle=\color{blue},
    numberstyle=\tiny\color{codegray},
    stringstyle=\color{codepurple},
    basicstyle=\ttfamily,
    breakatwhitespace=false,
    breaklines=true,
    captionpos=b,
    keepspaces=true,
    numbers=left,
    numbersep=5pt,
    showspaces=false,
    showstringspaces=false,
    showtabs=false,
    tabsize=2
}

\lstset{style=mystyle}
\usepackage[whole]{bxcjkjatype}
\usepackage{RyuOhTeX}

\usepackage{amsmath}
\begin{document}


\title{\RyuOhTeX:將棋棋譜與盤面製作\LaTeX套件}
\author{Ping-Chen (Quisette) Chung}

	\maketitle
	\tableofcontents

\setcounter{section}{-1}
\section{前言}
\newpage
\section{Koma commands}

\section{盤面生成}

\subsection{完整盤面}
欲使用整個棋盤的盤面,可輸入\verb|\shogiban{\hirate}|

以生成將棋棋盤。若是要生成空棋盤,可使用 \verb|\shogiban{}| 來生成。\\
\begin{figure}[h]
  \begin{minipage}[h]{0.5\linewidth}
    \centering
    \shogiban{}
    \caption{空棋盤}
    \label{fig:side:a}
  \end{minipage}%
  \begin{minipage}[h]{0.5\linewidth}
    \centering
    \shogiban{\hirate}
    \caption{初始盤面}
    \label{fig:side:b}
  \end{minipage}
\end{figure}\\
若要生成先後逆(後手)的盤面,可使用\verb|\gyakuban{}|生成。\\
請注意,於 \RyuOhTeX  1.0 版本中,先後逆盤面的座標順序尚未更改。
\begin{figure}[h]
    \centering
    \gyakuban{}
    \caption{先後逆盤面}
    \label{fig:my_label}
\end{figure}\\
亦可使用圖像:

\begin{figure}[h]
  \begin{minipage}[h]{0.5\linewidth}
    \centering
    \gazouka
  \shogiban{
\koma11\Ky \koma12\Gy \koma14\Fu \koma17\FU \koma19\KY
\koma23\Fu \koma27\Ki \koma32\NG \koma34\FU \koma45\KE
\koma51\Ki \koma53\FU \koma54\Fu \koma58\Ry \koma61\Hi
\koma67\KI \koma74\Fu \koma76\FU \koma77\GI \koma79\OU
\koma81\Ke \koma84\Fu \koma87\FU \koma89\KE \koma91\Ky
\koma93\Fu \koma97\FU \koma99\KY \mochigoma[2]\KA
\mochigoma\KI \mochigoma[2]\GI \mochigoma\Ke \mochigoma[6]\Fu
}
    \caption{define using \texttt{\gazouka}\\}
    \label{fig:side:a}
  \end{minipage}%
  \begin{minipage}[h]{0.5\linewidth}
    \centering
    \def\komapath{Portella}
  \gazouka
  \shogiban{
\koma11\Ky \koma12\Gy \koma14\Fu \koma17\FU \koma19\KY
\koma23\Fu \koma27\Ki \koma32\NG \koma34\FU \koma45\KE
\koma51\Ki \koma53\FU \koma54\Fu \koma58\Ry \koma61\Hi
\koma67\KI \koma74\Fu \koma76\FU \koma77\GI \koma79\OU
\koma81\Ke \koma84\Fu \koma87\FU \koma89\KE \koma91\Ky
\koma93\Fu \koma97\FU \koma99\KY \mochigoma[2]\KA
\mochigoma\KI \mochigoma[2]\GI \mochigoma\Ke \mochigoma[6]\Fu
}
    \caption{Self-Defined Piece Image}
    \label{fig:side:b}
  \end{minipage}
\end{figure}



\subsection{詰將棋(局部)盤面}
使用指令
\begin{lstlisting}
\tsumeshogi{#1}{#2}{#3}{#4}{
#5
}
\end{lstlisting}
分別指定限制的直行(筋)從第 \#1 筋到第 \#2筋為止;\\
限制的橫列(段)從第 \#3 段到第 \#4段為止。\\
\#5 放置所有使用到的棋駒。
\begin{lstlisting}
\tsumeshogi{1}{4}{1}{5}{
    \koma11\Ky \koma13\Fu \koma22\Gy
    \koma23\KA \koma42\Fu
    \mochigoma\HI\mochigoma[2]\KI
}
\end{lstlisting}

\begin{figure}[h]
  \begin{minipage}[h]{0.5\linewidth}
    \centering
   \tsumeshogi{1}{4}{1}{5}{
    \koma11\Ky \koma13\Fu \koma22\Gy
    \koma23\KA \koma42\Fu
    \mochigoma\HI\mochigoma[2]\KI
    }
    \caption{詰將棋範例}
    \label{fig:side:a}
  \end{minipage}%
  \begin{minipage}[h]{0.5\linewidth}
    \centering
   \tsumeshogi{1}{5}{6}{9}{
    \koma16\FU \koma19\KY \koma27\FU \koma28\GY \koma29\KE \koma37\FU \koma47\FU\koma57\FU\koma38 \GI \koma49 \KI \koma58\KI
    }
    \caption{美濃圍範例}
    \label{fig:side:b}
  \end{minipage}
\end{figure}

\newpage
\section{棋譜生成}


\kuro76\FU \siro34\FU \kuro88\KA\naru \siro00\dou\GI\\



\begin{lstlisting}
\kuro76\FU \siro34\FU \kuro88\KA\naru \siro\dou\GI
\end{lstlisting}

\section{指令與規則}

\section{Included Graphics Path}


\begin{figure}[h]
  \centering
  \gazouka
  \gyakuban{
    \hirate
  }
  \caption{shogiban}
  \end{figure}

  
\begin{figure}[h]
    \centering
    % \gazouka
\scalebox{1.5}[1.5]{

\tsumeshogi{1}{4}{1}{5}{
        \koma11\NK\koma12\Nk
        \koma21\NG\koma22\Ng
        \koma31\NY\koma32\Ny
    }

}
    \caption{shogiban}
\end{figure}




\end{document}


\section{盤面生成}

\subsection{完整盤面}
\subsubsection{一般文字盤面}
欲使用整個棋盤的盤面,可輸入\verb|\shogiban{\hirate}|以生成將棋棋盤。若是要生成空棋盤,可使用 \verb|\shogiban{}| 來生成。\\
\begin{figure}[h]
  \begin{minipage}[h]{0.5\linewidth}
    \centering
    \shogiban{}
    \caption{空棋盤}
    \label{fig:side:a}
  \end{minipage}%
  \begin{minipage}[h]{0.5\linewidth}
    \centering
    \shogiban{\hirate}
    \caption{初始盤面}
    \label{fig:side:b}
  \end{minipage}
\end{figure}\\
\subsubsection{棋駒爲圖片的盤面}

使用者可使用\verb|\shogiban{\gazouka <棋駒位置>}| 以用棋駒圖像顯示盤面,且可用\verb|\def\komapath{<棋駒圖片位置>}|指定棋駒圖片的路徑。
棋駒圖片的命名方式可參照本資料夾目錄。
\begin{figure}[h]
  \begin{minipage}[h]{0.5\linewidth}
    \centering
    
  \shogiban{\gazouka
\koma11\Ky \koma12\Gy \koma14\Fu \koma17\FU \koma19\KY
\koma23\Fu \koma27\Ki \koma32\NG \koma34\FU \koma45\KE
\koma51\Ki \koma53\FU \koma54\Fu \koma58\Ry \koma61\Hi
\koma67\KI \koma74\Fu \koma76\FU \koma77\GI \koma79\OU
\koma81\Ke \koma84\Fu \koma87\FU \koma89\KE \koma91\Ky
\koma93\Fu \koma97\FU \koma99\KY \mochigoma[2]\KA
\mochigoma\KI \mochigoma[2]\GI \mochigoma\Ke \mochigoma[6]\Fu
}
    \caption{使用棋駒圖片顯示的棋盤}
    \label{fig:side:a}
  \end{minipage}%
  \begin{minipage}[h]{0.5\linewidth}
    \centering
    \def\komapath{Portella}
  \gazouka
  \shogiban{
\koma11\Ky \koma12\Gy \koma14\Fu \koma17\FU \koma19\KY
\koma23\Fu \koma27\Ki \koma32\NG \koma34\FU \koma45\KE
\koma51\Ki \koma53\FU \koma54\Fu \koma58\Ry \koma61\Hi
\koma67\KI \koma74\Fu \koma76\FU \koma77\GI \koma79\OU
\koma81\Ke \koma84\Fu \koma87\FU \koma89\KE \koma91\Ky
\koma93\Fu \koma97\FU \koma99\KY \mochigoma[2]\KA
\mochigoma\KI \mochigoma[2]\GI \mochigoma\Ke \mochigoma[6]\Fu
}
    \caption{自定義棋駒圖片之棋盤}
    \label{fig:side:b}
  \end{minipage}
\end{figure}
\subsubsection{先後逆盤面}
若要生成先後逆(後手)的盤面,可使用\verb|\gyakuban{}|生成。\\
\begin{figure}[h]
    \centering
    \gyakuban{
    \koma88\Ka
    }
    \caption{先後逆盤面}
    \label{fig:my_label}
\end{figure}\\

\subsection{局部盤面}
使用指令
\begin{lstlisting}
\tsumeshogi{#1}{#2}{#3}{#4}{
#5
}
\end{lstlisting}
分別指定限制的直行(筋)從第 \#1 筋到第 \#2筋為止;\\
限制的橫列(段)從第 \#3 段到第 \#4段為止。\\
\#5 放置所有使用到的棋駒。
\begin{lstlisting}
\tsumeshogi{1}{4}{1}{5}{
    \koma11\Ky \koma13\Fu \koma22\Gy
    \koma23\KA \koma42\Fu
    \mochigoma\HI\mochigoma[2]\KI
}
\end{lstlisting}

\begin{figure}[h]
  \begin{minipage}[h]{0.5\linewidth}
    \centering
   \tsumeshogi{1}{4}{1}{5}{
    \koma11\Ky \koma13\Fu \koma22\Gy
    \koma23\KA \koma42\Fu
    \mochigoma\HI\mochigoma[2]\KI
    }
    \caption{詰將棋範例}
    \label{fig:side:a}
  \end{minipage}%
  \begin{minipage}[h]{0.5\linewidth}
    \centering
   \tsumeshogi{1}{5}{6}{9}{
    \koma16\FU \koma19\KY \koma27\FU \koma28\GY \koma29\KE \koma37\FU \koma47\FU\koma57\FU\koma38 \GI \koma49 \KI \koma58\KI
    }
    \caption{美濃圍範例}
    \label{fig:side:b}
  \end{minipage}
\end{figure}

\newpage